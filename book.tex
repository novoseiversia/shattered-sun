% Copyright (C) 2024 Nile Jocson <novoseiversia@gmail.com>

\documentclass{book}



\title{Shattered Sun}
\author{Nile Jocson}



\begin{document}
	\frontmatter{}
		\maketitle{}

		Copyright \copyright{}~2024 Nile Jocson \textless{}novoseiversia@gmail.com\textgreater{}.



	\mainmatter{}
		\chapter*{Prologue}
			4.6 billion years ago, primordial hydrogen and other elements synthesized by the first
			20 minutes of the Big Bang formed a nebula, 65 light years in diameter. A fragment of
			this molecular cloud moved into the center, collapsing under its own gravity into the
			star that we now call the Sun. The rest of the mass collected into a disk, forming the
			Earth and the other planets. This is the formation of the Solar System.

			The Sun is a massive ball of plasma, consisting mainly of hydrogen, helium, and some
			other heavier elements. Like all stars, it radiates away energy in the form of heat and
			light from the nuclear fusion that happens in its core, transforming hydrogen into
			helium. The Sun will become a red giant after it exhausts the hydrogen in its core,
			soon fusing the helium into carbon instead, expanding and being more unstable in the
			process. Finally, it will eject its outer layers, exposing the core inside. This core
			will stop shining after trillions of years, becoming a black dwarf; an iron shell of
			what the Sun had once been.

			Stars much larger than the Sun however, do not fade into obscurity at the end of their
			lives. Instead, they explode into a supernova, an event even brighter than that of an
			entire galaxy; and then they collapse into a black hole, an object so dense that
			anything that falls into it is completely stretched thin, like pasta, in a process
			called spaghettification. Black holes are completely dark and unseeable; they absorb
			all light and emit nothing back. We can only detect them because of how they affect
			their surroundings.

			When a star closely passes a supermassive black hole, like Sagittarius A*, the black
			hole in the center of our Milky Way galaxy, the gravity of the black hole is so strong
			that it pulls apart a spaghettified stream of matter from the star into its orbit; this
			is called a tidal disruption event. If the supermassive black hole eats up the star, by
			absorbing the matter in orbit after a tidal disruption event, or by simply swallowing
			the star whole, the black hole is given a new name: `quasar'

			Astrophysicists have been extremely fascinated by quasars ever since their discovery in
			the 1950s. Quasars are beautiful, bright, and dangerous; they are one of the most
			luminous phenomena in our universe, with an energy output orders of magnitude higher
			than that of our Milky Way, which contains 400 billion stars. They radiate away this
			energy across all wavelengths, from radio waves to gamma rays; waves with frequencies
			even higher than that of X-rays.

			In the present day, scientists like me have been trying to figure out how to harness
			energy from space in order to counteract the dwindling resources here on Earth, while
			supporting all of humanities' energy-intensive infrastructure, which has ramped up in
			consumption in the previous century. The problem of energy has plagued the advancement
			of our technology, and caused humanity to adopt ineffective, inefficient and infeasible
			solutions, destroying the planet that we all live in, the planet that I love.

			I believe that the key to this unpickable lock lies in the quasar. If we somehow manage
			to harness the power of a quasar, we would have 100 duodecillion joules of energy per
			second at the tip of our fingertips. We would rule the galaxy with Earth at its center,
			just like how the supermassive black holes that make quasars remain in complete
			gravitational control of entire galaxies. We would have enough energy to sustain the
			conquest of humanity until the heat death of the universe.

			And I believe, that with my blood, sweat and tears, I have finally forged this key for
			humanity to use.



\end{document}
