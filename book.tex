% Copyright (C) 2024 Nile Jocson <novoseiversia@gmail.com>

\documentclass{book}



\newcommand{\spacer}{\[\]}



\title{Shattered Sun}
\author{Nile Jocson}



\begin{document}
	\frontmatter{}
		\maketitle{}

		Copyright \copyright{}~2024 Nile Jocson \textless{}novoseiversia@gmail.com\textgreater{}.



	\mainmatter{}
		\chapter*{Prologue}
			4.6 billion years ago, primordial hydrogen and other elements synthesized by the first
			20 minutes of the Big Bang formed a nebula, 65 light years in diameter. A fragment of
			this molecular cloud moved into the center, collapsing under its own gravity into the
			star that we now call the Sun. The rest of the mass collected into a disk, forming the
			Earth and the other planets. This is the formation of the Solar System.

			The Sun is a massive ball of plasma, consisting mainly of hydrogen, helium, and some
			other heavier elements. Like all stars, it radiates away energy in the form of heat and
			light from the nuclear fusion that happens in its core, transforming hydrogen into
			helium. The Sun will become a red giant after it exhausts the hydrogen in its core,
			soon fusing the helium into carbon instead, expanding and being more unstable in the
			process. Finally, it will eject its outer layers, exposing the core inside. This core
			will stop shining after trillions of years, becoming a black dwarf; an iron shell of
			what the Sun had once been.

			Stars much larger than the Sun however, do not fade into obscurity at the end of their
			lives. Instead, they explode into a supernova, an event even brighter than that of an
			entire galaxy; and then they collapse into a black hole, an object so dense that
			anything that falls into it is completely stretched thin, like pasta, in a process
			called spaghettification. Black holes are completely dark and unseeable; they absorb
			all light and emit nothing back. We can only detect them because of how they affect
			their surroundings.

			When a star closely passes a supermassive black hole, like Sagittarius A*, the black
			hole in the center of our Milky Way galaxy, the gravity of the black hole is so strong
			that it pulls apart a spaghettified stream of matter from the star into its orbit; this
			is called a tidal disruption event. If the supermassive black hole eats up the star, by
			absorbing the matter in orbit after a tidal disruption event, or by simply swallowing
			the star whole, the black hole is given a new name: `quasar'

			Astrophysicists have been extremely fascinated by quasars ever since their discovery in
			the 1950s. Quasars are beautiful, bright, and dangerous; they are one of the most
			luminous phenomena in our universe, with an energy output orders of magnitude higher
			than that of our Milky Way, which contains 400 billion stars. They radiate away this
			energy across all wavelengths, from radio waves to gamma rays; waves with frequencies
			even higher than that of X-rays.

			In the present day, scientists like me have been trying to figure out how to harness
			energy from space in order to counteract the dwindling resources here on Earth, while
			supporting all of humanities' energy-intensive infrastructure, which has ramped up in
			consumption in the previous century. The problem of energy has plagued the advancement
			of our technology, and caused humanity to adopt ineffective, inefficient and infeasible
			solutions, destroying the planet that we all live in, the planet that I love.

			I believe that the key to this unpickable lock lies in the quasar. If we somehow manage
			to harness the power of a quasar, we would have 100 duodecillion joules of energy per
			second at the tip of our fingertips. We would rule the galaxy with Earth at its center,
			just like how the supermassive black holes that make quasars remain in complete
			gravitational control of entire galaxies. We would have enough energy to sustain the
			conquest of humanity until the heat death of the universe.

			And I believe, that with my blood, sweat and tears, I have finally forged this key for
			humanity to use.



		\chapter{Luminescence}
			The Earth is a polluted and disgusting mess. It's been scarred by centuries of misuse
			and abuse at the hands of humanity. All of its natural resources have been exhausted;
			rivers trashed, lakes dried up, forests shaved clean and mountains bored completely
			through. But there are still some places of relative paradise here in the desolate
			Terran wasteland; I, for one, live in a particularly peaceful prairie in the middle of
			what had once been the Sino-Russian border together with my husband. While the
			lithosphere on Earth has been thoroughly ruined, the sky still remains clear,
			unscorched and uncaring of the events of the past millenia. Looking up, I am still
			bombarded by the blue tint stretching from horizon to horizon, I can still pretend like
			I could eat the clouds above that look like tasty white cotton candy, and I can still
			feel the rays of the Sun absorb on my cheeks as I wake up in my window-side bed in the
			morning. It's a wonder how the sky could live on as the beautiful thing that it was
			centuries ago; my great-grandparents, their children, and their children's children all
			remember the sky in the same way, because the sky looked the same to all of us.

			Nothing compares to the beauty of the Earth; not the luminous white flowers that sprout
			out of the pitch black soil in Eclipta, nor the saturated mesa-like color bands of the
			mountains in Geraea. I've been to every single planet that humanity has colonized ever
			since we perfected faster-than-light galactic travel, and even though their features
			have remained unaffected by the destructive gaze of humans, I wouldn't look at them the
			same way as I look at the Earth; my home.

			This patch of land that my family has been living in was gifted to us by the Galactic
			Council, the administration that governs over each and every planet that humanity has
			spread to. I was the proponent of capturing relativistic jets from far away quasars as
			a source of energy, the technology that we now call `relative capture'. Relative
			capture has removed the barrier of limited energy in research and technology, which has
			allowed us to funnel gigantic amounts of power into systems without wasting a single
			gram of a planet's natural resources. Without relative capture, humanity would not be
			colonizing and naming thousands of planets after plant genera; without relative
			capture, humanity would not be able to terraform uninhabitable planets, or transmutate
			common materials into the rare elements that are only found in the dust clouds of
			supernovae.

			I had retired from astrophysics 5 years ago since my job had been fully done at this
			point; relative capture has been reaching its practical limits, and there is pretty
			much nothing that can be done to optimize it further. The energy of a relativistic jet
			from a quasar billions of light-years away would have redshifted by a lot before it
			reaches the Earth; nothing can be done about that unless nearer quasars just somehow
			come into existence. The only thing that relative capture actually does is redirect the
			jet into a collector; it cannot amplify the jet or create energy from something that
			isn't already there.

			I miss the life of a researcher, and I wish that there was more to do; but frankly, I
			enjoy resting in the countryside with my husband even more.

			``The night sky is so gorgeous today.'' I said as I lie down in the recently-cut grass,
			looking up at the black-blue sky. I give a short glance to my husband Rufus, wondering
			if he's also admiring it.

			``Indeed it is.'' he replies. ``It's also gorgeous how it's been years and you still
			make it a thing to always point it out.''

			``Indeed I do. Well, the thing is that it always is beautiful.''

			``Yes. But I'm reminded more of how fucked we are when I look at it.''

			``Why is that?''

			``We've been trying to darken the sky ever since the Industrial Revolution. We've
			made all the effort to deface it just because, yet the only thing that happened is
			that it became gray for like, a century. That's barely a blip in the whole scheme of
			things.''

			``If this is what you think about every single time that we've lied down here, then
			you'd be even more fucked than `we' are.'' I retort, jokingly.

			``But it's true, isn't it?''

			``I'd rather the sky didn't become gray for a century. If we'd continued down that
			path, it would've been a until the end of time instead.''

			``That's also true. Thank god, I guess.''

			``I guess?'' I exclaim in feigned disgust.

			``I'll get some more nachos.'' he said, giving me a sly smirk and gets up out of reach
			of my arms. ``You ate them all and left me nothing, you monster.''

			``You know I love nachos. And karma. Do get me some more though.''

			The widespread usage of relative capture as the primary source of energy on most
			planets have made the night even prettier; in my opinion, at least. Bright yellow
			streaks litter the sky; those streaks are the relativistic jets of different quasars
			being redirected into the planets governed by the Galactic Council. It's impossible to
			see but those streaks are curved with the radii of thousands of light years, exactly
			angled in order not to miss the collectors and wreak guaranteed havoc on what
			unprotected matter lies beneath. While yes, the energy would have dissipated from
			travelling in empty space for what used to be an unimaginable distance, the ionized
			matter would still be carrying Joules of energy in the range of decillions. For
			context, humanity, even now with our massive energy usage, have only used 521
			yottajoules of energy in total since the dawn of time. Total relative capture would
			provide way more than humanity would need beyond even the time where the universe will
			experience maximum entropy.

			Of course, no one needs this much energy, so relative capture is toned down by a lot.
			But in theory, it would be possible to achieve total relative capture with the machines
			that we use today. But imagine, imagine if the relativistic jet somehow misses the
			target collector and absorbs into the ground; what damage would be done to the exposed
			planet below? I pray that this scenario would never play out in real life, the way that
			it replays in my head every time I notice the magnificent yellow lines in the sky.

			``Sorry for the wait. I melted some cheese too. Leave some for me, okay? I made this
			cheese for me, not you. I'm merely letting you share.''

			He notices something on the corner of his eye, something that he hadn't seen before.
			He froze just above my field of vision; I look at where his eyes were pointing at, and
			a spectacular sight cuts all of my focus on the view of the Milky Way galaxy around it.

			\spacer{}

			A yellow beam of light.

			\spacer{}

			This beam of light isn't anything that I've seen before. The ray becomes more intense
			and its length growing with each second that passes. It curves around Sagittarius A*,
			revealing the position of the black hole that normally isn't visible to the naked eye.
			The beam weaves through stars like a line in a dense scatter plot; it intersects with
			many of the relativistic jets that are being absorbed by relative capture machines on
			other faraway planets. These things; the beam, the stars and the jets create a system
			of lines and points in the sky; a constellation where the lines are drawn out for you.
			Space has become the drawing board for our civilization, and the sky is the projection
			of it into two-dimensional space.

			``That is the most awesome thing that I've ever seen in my life.'' he says.

			I remain frozen in place on the ground, centering the growing end of the beam in my
			vision. I start to realize the disaster that is happening. This is a relativistic jet.
			But how? The Galactic Council always broadcasts the construction and commencement of
			new relative capture machines; and I am completely certain that there are no quasars
			that exist inside the Milky Way galaxy. Sagittarius A* used to be one, but that was
			6 million years ago. No, this relativistic jet came from somewhere else, but where?
			Humanity still hasn't developed the technology to just randomly fire jets into space,
			especially not a jet like this that is brighter than the entire galaxy that it imposes
			on.

			``Helia, are you okay?''

			I don't reply, and I don't think I even noticed him speak. I start to panic, thinking
			about the possibilities of what has gone wrong in this exact moment. The scenario I've
			been thinking about over and over in my head is utterly unbearable; the destruction of
			a planet or even just the lasering of a city is unimaginable. The consequences would be
			extreme, for our civilization as a whole. How would all my failsafes fail? All my
			safety nets cut? Who would do such a thing, or rather, how did I fuck up this badly?
			The anxiety of what may have been happening was starting to make me combust like the
			hypergolic propellants in early unstable rockets; until the phone rang.



		\chapter{Oblivia}



\end{document}
