% Copyright (C) 2024 Nile Jocson <novoseiversia@gmail.com>

\documentclass{book}

\usepackage{musicography}



\newcommand{\spacer}{\[\]}
\newcommand{\timeskip}{\begin{center}\musSegno{}\end{center}}



\title{Shattered Sun}
\author{Nile Jocson}



\begin{document}
	\frontmatter{}
		\maketitle{}

		Copyright \copyright{}~2024 Nile Jocson \textless{}novoseiversia@gmail.com\textgreater{}.



	\mainmatter{}
		\chapter*{Prologue}
			When I was sent to stay in Astera-232 for the first time, I got a glimpse of space and
			its vast empty nothingness. Nothingness, because none of it was reachable to humanity,
			and thus, space was completely useless to us. Everything is just too far apart, and we
			didn't have the technology to explore all of its features yet. I was sitting in the
			observation deck of the ship, looking out into the ocean of star systems, galaxies, and
			nebulae. While we didn't have the technology to travel to them, we did have the technology
			to observe them in their full beauty from far far away; field magnifiers enabled us to
			discern if there was indeed life in the Earth-like exoplanets, and back then it let me
			look at the deep gray craters of Pyrum, the eruption-like coronal mass ejections of
			HST-156e5, and the curved light coming from behind Gigantua, a supermassive black hole.

			But something else had caught my eye; something that I've only heard of, and had never
			seen before. An enormous disk of gas and space dust spinning around a fixed point at
			unimaginable speeds; it looked like a spiraling flame, burning with the color and fluidity
			of an aurora borealis. A jet of ionized matter shooting out from the center of it, so bright
			and blue that it looked like an intergalactic laser; but I could see each wave, each pulse
			and cloud of space stuff being launched at near light-speed into deep space. And at its
			center, the absolute dark; the blackest thing in the universe, so dense and inescapable,
			that even light is indefinitely jailed inside of it.

			This was a quasar. The brightest thing in the universe, being powered by the darkest thing in
			the universe; a metaphorical yin-yang, and the greatest paradox I have ever set my gaze
			upon. This was the most beautiful thing I've ever seen.

			Quasars gave me a new hope for humanity, a new hope in the science of celestial phenomena;
			a new hope that we would finally escape the gravity well of our own Sun, and transcend the
			shackles of our Solar System prison. I'm sick of us, sick of the fact that we have to deface
			our own planet just to survive. All this technology to make our lives better, but at what
			cost? Earth has been suffering, and humanity has been too. The irony is not lost on me; that
			to make so much effort to better our lives, has only sent it deeper into hell.

			The unpickable lock of energy lies in the quasar. I'm certain of it. If we somehow managed to
			harness its power, we would have 100 duodecillion joules of energy per second, at the very tip of
			our fingertips. We would rule the galaxy with Earth at its center, just like how the
			supermassive black hole in the center of quasars remain in complete gravitational control
			of entire galaxies. We would have enough energy to sustain the conquest of humanity until the
			last breath of the universe. Earth would finally be spared from our destruction.

			Today, the key to this unpickable lock has been forged. By my blood, sweat, and tears,
			I have made the magnificent quasar come into life.



		\chapter{Luminescence}
			The Earth is a polluted and disgusting mess. It's been scarred by centuries of misuse
			and abuse at the hands of humanity. All of its natural resources have been exhausted;
			rivers trashed, lakes dried up, forests shaven clean and mountains bored completely
			through. But there are still some places of relative paradise here in the desolate
			Terran wasteland; I, for one, live in a particularly peaceful prairie in the middle of
			what had once been the Chinese border together with my husband. While the
			lithosphere on Earth has been thoroughly ruined, the sky still remains clear,
			unscorched and uncaring of the events of the past millenia. Looking up, I am still
			bombarded by the blue tint stretching from horizon to horizon, I can still pretend like
			I could eat the clouds above that look like tasty white cotton candy, and I can still
			feel the rays of the Sun absorb on my cheeks as I wake up in my window-side bed in the
			morning. It's a wonder how the sky could live on as the beautiful thing that it was
			centuries ago; my great-grandparents, their children, and their children's children all
			remember the sky in the same way, because the sky looked the same to all of us.

			Nothing compares to the beauty of the Earth; not the luminous white flowers that sprout
			out of the pitch black soil in Eclipta, nor the saturated mesa-like color bands of the
			mountains in Geraea. I've been to every single planet that humanity has colonized ever
			since we perfected faster-than-light galactic travel, and even though their features
			have remained unaffected by the destructive gaze of humans, I wouldn't look at them the
			same way as I look at the Earth; my home.

			This patch of land that my family has been living in was gifted to us by the Galactic
			Council, the administration that governs over each and every planet that humanity has
			spread to. Before this, we lived in the bustling city of Bacoor in the Philippines, but
			no one has been allowed to stay in the country anymore ever since other planets have been
			colonized, and all Filipinos have been forced to move to other planets. We're lucky that we
			get to just stay here on Earth; right now, the only people allowed here are the people
			that were employed to rehabilitate the land and the waters.

			I was the proponent of the jets of ionized matter from far away quasars as
			a source of energy, the technology that we now call `relative capture'. Relative
			capture has removed the barrier of limited energy in research and technology, which has
			allowed us to funnel gigantic amounts of power into our systems without wasting a single
			gram of a planet's natural resources. Without relative capture, humanity would not be
			colonizing and naming thousands of planets after plant genera; without relative
			capture, humanity would not be able to terraform uninhabitable planets, or be able to
			transmutate common materials into the rare elements that are only found in the dust clouds of
			supernovae.

			I had retired from astrophysics 5 years ago because my job had been fully done at this
			point; relative capture has been reaching its practical limits, and there is pretty
			much nothing that can be done to optimize it further. The energy of a relativistic jet
			from a quasar billions of light-years away would have redshifted by a lot before it
			reaches the Earth, and nothing can be done about that, unless nearer quasars just somehow
			spawn into existence. The only thing that relative capture actually does is redirect the
			jet into a collector; it cannot amplify the jet or create energy from something that
			isn't already there.

			I miss the life of a researcher, and I wish that there was more to do; but frankly, I
			enjoy resting in the countryside with my husband even more.

			``The night sky is so gorgeous today.'' I said as I lie down in the recently-cut grass,
			looking up at the black-blue sky. I give a short glance to my husband Seraph, wondering
			if he's also admiring it.

			``Indeed it is.'' he replies. ``It's also gorgeous how it's been years and you still
			make it a thing to always point it out.''

			``Indeed I do. Well, the thing is that it always is beautiful.''

			``Yes. But I'm reminded more of how fucked we are when I look at it.''

			``Why is that?''

			``We've been trying to darken the sky ever since the Industrial Revolution. We've
			made all the effort to deface it just because, yet the only thing that happened is
			that it became gray for like, a century. That's barely a blip in the whole scheme of
			things.''

			``If this is what you think about every single time that we've lied down here, then
			you'd be even more fucked than `we' are.'' I retort, jokingly.

			``But it's true, isn't it?''

			``I'd rather the sky didn't become gray for a century. If we'd continued down that
			path, it would've been like that until the end of time instead.''

			``That's also true. Thank god, I guess.''

			``I guess?'' I exclaim in feigned disgust.

			``I'll get some more nachos.'' he said, giving me a sly smirk and gets up out of reach
			of my arms. ``You ate them all and left me nothing, you monster.''

			``You know I love nachos. And karma. Do get me some more though.''

			The widespread usage of relative capture as the primary source of energy on most
			planets have made the night even prettier; in my opinion, at least. Bright yellow
			streaks litter the sky; those streaks are of the relativistic jets of different quasars
			being redirected into the planets governed by the Galactic Council. It's impossible to
			see, but those streaks are curved with the radii of thousands of light years, angled incredibly
			precisely, in order not to miss the collectors and wreak guaranteed havoc on what
			unprotected matter lies beneath. While yes, the energy would have dissipated from
			travelling in empty space for what used to be an unimaginable distance, the ionized
			matter would still be carrying joules of energy in the range of decillions. For
			context, humanity, even now with our massive energy usage, have only used 521
			yottajoules of energy in total since the dawn of time. Total relative capture would
			provide way more than that in a single second, and what it could generate in this singular
			second would last humanity even beyond the final heat death of the universe.

			Of course, no one needs this much energy, so relative capture is designed to redirect
			only a small part of the jet, so that only a percentage of the energy is harnessed.
			But in theory, it would be possible to achieve total relative capture with the machines
			that we use today. But imagine, imagine if the relativistic jet somehow misses the
			target collector and absorbs into the ground; what damage would be done to the exposed
			planet below? I pray that this scenario would never play out in real life, the way that
			it plays out in my head every time I notice the magnificent yellow lines in the sky.

			``Sorry for the wait. I melted some cheese too. Leave some for me, okay? I made this
			cheese for me, not you. I'm merely letting you share.''

			He notices something on the corner of his eye, something that he'd never seen before.
			He froze just above my field of vision; I look at where his eyes were pointing at, and
			a spectacular sight cuts all of my focus on the view of the Milky Way galaxy around it.

			A yellow beam of light.

			This beam of light isn't anything that I've seen before; I could tell that this was from
			very far away, and yet, it puts all the lights in the night sky to shame. The around it
			are completely drowned out by its intense glow, and it looks like its swallowing more
			stars as the beam travels space. I forget about everything for a second, as I admire the
			incredible sight.

			``That is the most awesome thing that I've ever seen in my life.'' he says.

			I remain frozen in place on the ground, wondering what this beam of light may have been.
			Then it hits me. I stand up and rush inside the house to get my personal field magnifiers,
			and I look at the enlarged beam in order to see its details. My worst suspicions were
			confirmed. The pulsing and the speed of the beam were unmistakable, and its color was
			textbook; this was a relativistic jet. But this was different; this jet was setting the
			area around it on fire with how potent it is. This was orders of magnitude more powerful
			than the jets that we use in relative capture, and the disaster that would happen if this
			hits a planet would be unimaginable. My face grows red.

			``Helia, are you okay?''

			I don't reply. My hearing has blurred from my anxiety, and his voice now just sounds like
			garbling to me. I start to panic, thinking about what exactly had gone wrong in this moment.
			This is the same as the unbearable scenario that had been replaying in my head over and over again,
			and to say the least, this is not good. The destruction of a planet or even just the lasering of a
			city is unimaginable, and the consequences would be extreme, for our civilation as a whole. How
			did all my failsafes fail? All my safety nets cut? How could I fuck up this badly? My mind
			started spiraling, like how the cloud of gas and dust spirals around a quasar, and right now, I am
			a particle in that dust cloud, getting dizzier and dizzier by the second.

			And then the phone rang.
\end{document}
